\documentclass{article}
\usepackage[utf8]{inputenc}
\usepackage{amsmath}

\title{LA Lecture 7}
\author{swati.allabadi }
\date{}

\begin{document}

\maketitle

\section{Topics Covered}
\begin{enumerate}
    \item Computing the null space (Ax = 0)
    \item Pivot variables-free variables
    \item Special solutions-rref(A) = R
    
\end{enumerate}
\textbf{Rank} of a matrix = Number of pivots in that matrix. 
\subsection{Algorithm to obtain null space of a matrix }
Note that the given matrix (A) can be a rectangular matrix.
Steps of the algorithm are as follows:
\begin{enumerate}
    \item Apply elimination procedure on the matrix A. The form of the matrix obtained through this procedure is termed as Echelon form.
    \item Obtain Row Reduced Echelon form (rref) through this matrix.
    \item The null space is all the linear combinations of the columns of matrix \[ \begin{bmatrix}
    -F\\
    I\\
    \end{bmatrix} \]
    
    where I is the identity matrix and F is the coefficients of free variables in the rref.
    \newline
    Number of columns in this matrix = Number of free  variables.
\end{enumerate}
Some key point regarding the algorithm :
\begin{enumerate}
    \item In step 1 of the algo, we don't have to do  row exchanges even if we get 0's in the pivot positions.
    \textbf{Verify this for some other example.}
    \item With elimination we are not changing the null space. Since the solution to Ax = 0 won't change with it as the same steps are applied on the RHS.
    What changes with elimination is the column space.
    \item The columns containing the pivot are termed as \textbf{pivot columns} and the corresponding variables are termed as \textbf{pivot variables} while the remaining columns are termed as \textbf{free columns} and the corresponding variables are termed as \textbf{free variables}. They are termed as free because we can assign any value to the variables associated with those columns. And then solve for the variables associated with pivot variables.
    
\end{enumerate}
Let's apply this algorithm on an instance.
Let A = \[ 
        \begin{bmatrix}
        1 & 2 & 2 & 2\\
        2 & 4 & 6 & 8\\
        3 & 6 & 8 & 10\\
        \end{bmatrix}
        \]

Applying step $E_{21}$ and $E_{31}$ on A we get :
\[ 
        \begin{bmatrix}
        1 & 2 & 2 & 2\\
        0 & 0 & 2 & 4\\
        0 & 0 & 2 & 4 \\
        \end{bmatrix}
        \]
        
The first pivot is at $A_{11}$ and second is at $A_{23}$. Thus we do $E_{33}$ in the next step and get :

        \begin{bmatrix}
        1 & 2 & 2 & 2\\
        0 & 0 & 2 & 4\\
        0 & 0 & 0 & 0 \\
        \end{bmatrix}
         \textbf{= U}, which is called the \textbf{echelon form}. 
Clearly, the number of pivots in this matrix = 2, which is called the \textbf{rank}  of the matrix. \newline
Ux = 0 looks like :\\~\\
$x_1 + 2x_2 + 2x_3 + 2x_4 = 0$ \newline
$2x_3 + 4x_4 = 0$ \newline
Let's try to obtain null space using Ux = 0: \newline
Since $x_2$ and $x_4$ are free variables we can assign any values to them. Let's assign 1 and 0 to them respectively. We get  : \\~\\
$x_1 + 2 + 2x_3 = 0$ \newline
$2x_3 = 0$ \newline
It implies $x_3$ = 0 and $x_1$ = -2. 
\newline
Now, let's assign 0 and 1 to $x_2$ and $x_4$ respectively. We get  : \\~\\
$x_1 + 2x_3 + 2 = 0$ \newline
$2x_3 + 4 = 0$ \newline
It implies $x_3$ =-2 and $x_1 $ = 2.

\newline
What we get is 

c\begin{bmatrix}
-2 \\ 1 \\ 0 \\0
\end{bmatrix}  and
d\begin{bmatrix}
2 \\ 0 \\ -2 \\1
\end{bmatrix} are solutions of Ax = 0 or Ux = 0 where c and d are any constants.
\\~\\
Thus, the linear combinations of columns of 
\begin{bmatrix}
-2 & 2\\ 
1  & 0\\
0  & -2 \\
0  & 1
\end{bmatrix} is the null space of A.
\newline
Let us obtain row reduced echelon form of the above matrix. In this we try to get zeroes above pivot positions as well and make pivot elements = 1.
\newline
Performing $E_{13}$ on the above matrix we get: \newline
\[ 
        \begin{bmatrix}
        1 & 2 & 0 & -2\\
        0 & 0 & 2 & 4\\
        0 & 0 & 0 & 0 \\
        \end{bmatrix}
        \]
        Making pivot at $A_{23}$ = 1 we get
         \[ 
        \begin{bmatrix}
        1 & 2 & 0 & -2\\
        0 & 0 & 1 & 2\\
        0 & 0 & 0 & 0 \\
        \end{bmatrix}
        \] = R (in rref)
        The solutions to Ax = 0, Ux = 0 and Rx = 0 are all the same. The equations obtained from Rx = 0 are as : \newline
$x_1 + 2x_2  - 2x_4 = 0$ \newline
$1x_3 + 2x_4 = 0$ \newline
Let's look at the coefficients in a rearranged manner. \newline

\[ 
        \begin{bmatrix}
        1 & 0 &  & &  &2 & -2\\
        0 & 1 &  & &  & 0 &  2\\
        Identity \ part &&&&&  Free\ part\\
        
        \end{bmatrix}
        \]
        
If we write in the form,  \begin{bmatrix}
    -F\\
    I\\
    \end{bmatrix} , we get
     \begin{bmatrix}
    -2 & 2\\
    0 & -2\\
    1 & 0\\
    0 & 1\\
    
    \end{bmatrix}  which is same as the columns of the matrix obtained above (in a reordered form) as the solution of null space of A.
    \newline
    What do we get as null space of $A^T$? What is the rank of $A^T$?
    \textbf{complete notes and verify this lecture from book}

 
\end{document}

