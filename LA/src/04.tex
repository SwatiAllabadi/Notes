\documentclass{article}
\usepackage[utf8]{inputenc}

\usepackage{amsmath}

\title{LA Lecture Notes}
\author{swati.allabadi }


\begin{document}

\maketitle

\section{Lecture 4}
\subscetion{Topics Covered}
\begin{enumerate}
    \item Inverse of AB, $A^T$
    \item Product of elimination matrices
    \item A = LU (considering U is obtained without any row changes)
    
    
\subsection{Inverse of AB}

Consider A and B to be square invertible matrices, then what is the inverse of AB i.e. $(AB)^{-1} = ?$
\\~\\
Now, $B^{-1}A^{-1}AB  = I $. Also, $ABB^{-1}A^{-1} = I$. 
\newline
$\Rightarrow B^{-1}A^{-1}$ is the inverse of $AB$.
\newline
Considering A and B to be square matrices, implies their left and right inverses are same. 

\subsection{Inverse of $A^T$}
We know that $AA^{-1} = I$. Taking transpose on both sides we get,
\newline
$(A^{-1})^{T}A^T = I$.
$\Rightarrow (A^T)^{-1} = (A^{-1})^{T} $.
\newline
Hence, for a single matrix transpose and inverse can be done in any order.

\subsection{Product of elimination matrices (obtained without row changes)}
From previous lectures, $EA = U $ where A is the matrix obtained from system of equations to be solved, E is elimination matrix and U is upper triangular matrix. We also have, $ A = LU$ where L is lower triangular matrix. Hence, $L = E^{-1}$.
\newline
A = \[ \begin{bmatrix}
2 & 1\\
8 & 7\\

\end{bmatrix}
\]
\newline
For this A, we get E = \[ \begin{bmatrix}
1 & 0\\
-4 & 1\\
\end{bmatrix}
\] and U = \[ \begin{bmatrix}
2 & 1\\
0 & 3\\
\end{bmatrix}
\]

For this A and U, L = \[ \begin{bmatrix}
1 & 0\\
4 & 1\\
\end{bmatrix}
\]
There is no significant difference between E and L i.e. there is no evident reason why we should prefer L over E (A = LU over EA = U).
But it is there in a 3*3 matrix.
For a 3*3 matrix, $E_{32}E_{31}E_{21}A = U$ is the equation. 
Let $E_{31}$ be an identity matrix ,  $E_{32}$ = \[ \begin{bmatrix}
1 & 0 & 0\\
0 & 1 & 0\\
0 & -5 & 1\\
\end{bmatrix}
\] and 
$E_{21}$ = \[
\begin{bmatrix}
1 & 0 & 0\\
-2 & 1 & 0\\
0 & 0 & 1\\
\end{bmatrix}
\]
$E_{32}E_{31}E_{21}$ =  $E_{32}E_{21}$ = \[
\begin{bmatrix}
1 & 0 & 0\\
-2 & 1 & 0\\
10 & -5 & 1\\
\end{bmatrix}
\]
Now $L = E^{-1}$ = $E_{21}^{-1}E_{32}^{-1}$ 
\newline
We have $E_{21}^{-1}$ = \[
\begin{bmatrix}
1 & 0 & 0\\
2 & 1 & 0\\
0 & 0 & 1\\
\end{bmatrix}
\] and $E_{32}^{-1}$  = \[ \begin{bmatrix}
1 & 0 & 0\\
0 & 1 & 0\\
0 & 5 & 1\\
\end{bmatrix}
\]
 $L = E^{-1}$ = $E_{21}^{-1}E_{32}^{-1}$ = \[
\begin{bmatrix}
1 & 0 & 0\\
2 & 1 & 0\\
0 & 5 & 1\\
\end{bmatrix}
\]
Hence, if there are no row exchanges the multiplier goes directly into L.
\newline
Note that L has 1's on its diagonal. We can also decompose U into 2 matrices to get 1's on its diagonal.

\subsection{How many operations (multiplication and subtraction) on an n*n matrix A to get L and U?}

We need to perform $O(n)$ operations to make an element 0. Hence we perform $O(n^2)$ operations to get a pivot in a column. To be precise, we perform around $n^2 + (n-1)^2 + ... + 1^2 \approx 1/3n^2 = O(n^2)$ operations to get all the pivots.
\\~\\
Check if t should be $(n-1)^2 + ... + 1^2$ or $n^2 + (n-1)^2 + ... + 1^2$.


\subsection{If row changes required (Permutation matrix P)}
We can fix this with the help of permutation matrices. It's a matrix with the rows of identity matrix (in some order). Total no of permutation matrices for an n*n matrix is n!. We can call these n! matrices a group as the product of any 2 matrices of those is a matrix within those.  
Note that, $P^{-1} = P^T$
Hence, if row changes are required PA = LU. Thus we can first change the order of the rows and proceed to get L and U. 

\subsection{$R^TR$ and $RR^T$ are symmetric matrices where R is any rectangular matrix}
$(R^TR)^T = R^TR^{TT} = R^TR$ \newline
$(RR^T)^T = R^{TT}R^T = RR^T$


    
\end{enumerate}


\end{document}
