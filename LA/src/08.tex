\documentclass{article}
\usepackage[utf8]{inputenc}
\usepackage{amsmath}

\title{LA Lecture 8}
\author{swati.allabadi }
\date{}

\begin{document}

\maketitle

\section{Topics Covered}
\begin{enumerate}
    \item Complete solution of Ax = b
    \begin{enumerate}
        \item Identify if Ax = b has a solution.
        \item If it has a solution, how many solutions does it has?
        \item Find all the possible solution.
    \end{enumerate} 
\end{enumerate}

\subsection{Solvability condition for Ax = b to have a solution (or) Identify if Ax = b has a solution}

Let's begin with matrix A as used in previous lecture as example. The equations are as follows: \newline
$x_1 + 2x_2 + 2x_3 + 2x_4 = b_1$ \newline
$2x_1 + 4x_2 + 6x_3 + 8x_4 = b_2$ \newline
$3x_1 + 6x_2 + 8x_3 + 10x_4 = b_3$ \newline
\textbf{What should be the condition on $b_1, b_2\ and\ b_3$ for the above system of equations to have a solution?} The augmented matrix [A b] of the above system of equations is as follows: \newline
\[ 
        \begin{bmatrix}
        1 & 2 & 2 & 2 & b_1\\
        2 & 4 & 6 & 8 & b_2\\
        3 & 6 & 8 & 10 & b_3\\
        \end{bmatrix}
        \]
b has to be tagged along to track the changes in it while applying the elimination process. 
Applying $E_{21}$ and $E_{31}$ we get:
\[ 
        \begin{bmatrix}
        1 & 2 & 2 & 2 & b_1\\
        0 & 0 & 2 & 4 & b_2 - 2b_1\\
        0 & 0 & 2 & 4 & b_3 - 3b_1\\
        \end{bmatrix}
        \]
Applying $E_{33}$ we get:
\[ 
        \begin{bmatrix}
        1 & 2 & 2 & 2 & b_1\\
        0 & 0 & 2 & 4 & b_2 - 2b_1\\
        0 & 0 & 0 & 0 & b_3 - b_1 - b_2\\
        \end{bmatrix}
        \]
1st and 3rd columns are the pivot columns. \textbf{The condition on solvability is }
$b_3 - b_1 - b_2 = 0$ because the 3rd row has all zeroes in it. This condition was expected because in the matrix A, 3rd row = 1st row + 2nd row. \newline

\textbf{Solvability condition on b}\newline
\textbf{Ax = b is solvable if b lies in the column space of A, C(A). }\newline
Also, if there's a combination of rows of A which gives zero row, then the same combination of entries of b must give 0. \newline
\textbf{(Would not the first condition imply the 2nd? Would 2nd condition alone be sufficient?)}

\subsection{Algorithm to find the complete solution to Ax = b}
\begin{enumerate}
    \item To find $x_{particular}$: Set all the free variables = 0 and solve Ax = b for the pivot variables.
    \item Add $x_{null\ space}$ to $x_{particular}$.
    \item $x_{complete} = x_{particular} + x_{null\ space}$.
    
    \end{enumerate}
Let's apply the above algorithm to the previous example. 
\begin{enumerate}
    \item In the above eg, $x_2$ and $x_4$ are free variables. Putting them = 0, we get: \newline
    $x_1 + 2x_3 = 1$ and 
    $2x_3 = 3$.
    Solving these we get, $x_1$ = -2, $x_3$ = 3/2. Hence, 
    $x_p$ = \[
    \begin{bmatrix}
    -2\\
    0\\
    3/2\\
    0
    
    \end{bmatrix}
    \]
    \item The null space obtained for this A from previous lecture: 
    $x_{null\ space}$ = c\begin{bmatrix}
-2 \\ 1 \\ 0 \\0
\end{bmatrix}  +
d\begin{bmatrix}
2 \\ 0 \\ -2 \\1
\end{bmatrix}
where c and d are any constants.
\item $x_{complete}$ = 
\begin{bmatrix} 
-2 \\ 0 \\ 3/2 \\ 0
\end{bmatrix} +
c\begin{bmatrix}
-2 \\ 1 \\ 0 \\0
\end{bmatrix}  +
d\begin{bmatrix}
2 \\ 0 \\ -2 \\1
\end{bmatrix}
\end{enumerate}
We can't multiply $x_p$ by any constant as it won't be the solution any more. $x_n$ works fine because any vector in $x_n$ will give Ax = 0 only. 
Now, in $x_c$, $x_n$ is a subspace but $x_c$ is not a subspace because it goes through $x_p$ but not origin.
For our example, $x_c$ is a 2-D plane not passing through origin, a shifted plane.

\subsection{Determining number of solutions on the basis of relation between rank (r), m and n for an m*n matrix}
For an m*n matrix, $r \leq m$, $r \leq n$ because we can't have more than min(m,n) pivots. Basically, we can't have more than n pivots in  columns and m pivots in rows, hence we can't have more than min(m,n) pivots. \newline
Various cases are as follows:

\begin{enumerate}
    \item \textbf{Full column rank i.e. r = n, $n \leq m$}. r = n means there is pivot in very column hence, there is no free variable. \newline
    
    What will be the null space of such a matrix?
    It will only be the zero vector. N(A) = zero vector.
    \newline
    Solution to Ax = b? It's only $x_{particular}$ if it exists. 
    \textbf{Hence, it has either 0 or 1 solution.}
    
    \item \textbf{Full row rank i.e. r= m, $m \leq n$}. r = m means there is a pivot in every row. \newline
    
     \textbf{For every b, there exists a solution for Ax =b} (as we don't get any zero row in elimination).
     
     We have n-r (=n-m) free variables. 
    
    \item \textbf{r = m =n}. We have a square matrix with a pivot in every row (and column). It's a full rank matrix. \textbf{r = m = n means it is an invertible matrix. It's rref form is identity matrix. N(A) = zero vector. It always has a solution, there's no condition on b for Ax = b to have a solution.}
    
\end{enumerate}

Let us consider an example for each case mentioned above.

\begin{enumerate}
    \item  \textbf{r = n }\[
    \begin{bmatrix}
    1 & 3 \\
    2 & 1\\
    6 & 1\\
    5 & 1
    \end{bmatrix}
\] It's row reduced echelon form (rref) would be: \[
    \begin{bmatrix}
    1 & 0 \\
    0 & 1\\
    0 & 0\\
    0 & 0
    \end{bmatrix}
\] 

\item \textbf{r=m}.

\[
\begin{bmatrix}
1 & 2 & 6 & 5 \\
3 & 1 & 1 & 1
\end{bmatrix}
\]
It's rref = \[
\begin{bmatrix}
1 & 0 & - & - \\
0 & 1 & - & -
\end{bmatrix}
\]
- means there would be some numbers in that part. These numbers comprise the F part, free variables part.

\item 
\[
\begin{bmatrix}
1 & 2 \\
3 & 1 
\end{bmatrix}
\] It's rref = \[
\begin{bmatrix}
1 & 0\\
0 & 1 
\end{bmatrix}
\]
\end{enumerate}

\subsection{Summary} All the cases are summarised below:
\begin{enumerate}
    \item \textbf{r = m = n}: rref ( or R) = I. It has \textbf{1 solution}.
    
    \item \textbf{r = n $<$ m}: R = \begin{bmatrix}
    I\\
    0
    \end{bmatrix}
Here 0 means zero rows. It has \textbf{0 or 1 solution}.

\item \textbf{r =m $<$ n }: R = [I F]. F means free variables part. Pivots might not appear all in the initial columns, hence I would be scattered. For simplicity, we have written as it's all in the beginning.
It has \textbf{infinite solutions}.
\item \textbf{r $<$ m, r $<$ n}: R =
\begin{bmatrix}
I & F \\
0 & 0 \\
\end{bmatrix}
It has \textbf{0 or infinite solutions}.
\end{enumerate}

\end{document}
