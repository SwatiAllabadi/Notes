{\rtf1\ansi\ansicpg1252\cocoartf2577
\cocoatextscaling0\cocoaplatform0{\fonttbl\f0\fswiss\fcharset0 Helvetica;}
{\colortbl;\red255\green255\blue255;}
{\*\expandedcolortbl;;}
\paperw11900\paperh16840\margl1440\margr1440\vieww11520\viewh8400\viewkind0
\pard\tx566\tx1133\tx1700\tx2267\tx2834\tx3401\tx3968\tx4535\tx5102\tx5669\tx6236\tx6803\pardirnatural\partightenfactor0

\f0\fs24 \cf0 \\documentclass\{article\}\
\\usepackage[utf8]\{inputenc\}\
\\usepackage\{amsmath\}\
\
\\title\{LA Lecture 5\}\
\\author\{swati.allabadi \}\
\
\
\\begin\{document\}\
\
\\maketitle\
\
\\section\{Topics Covered\}\
\\begin\{enumerate\}\
    \\item  PA = LU (Section 2.7) (covered in notes of lecture 4)\
    \\item  Vector spaces and sub spaces (Section 3.1) \
\\end\{enumerate\}\
\
\\subsection\{Vector Spaces\}\
Examples of vectors spaces: \\newline\
$R^2$ = all real valued 2-dimensional vectors (or say) all vectors with 2 components (or)  the x-y plane \\newline\
Similarly, $R^n$ = all column vectors with n components\
\\[\
\\begin\{bmatrix\}\
3 \\\\\
-2\\\\\
\\end\{bmatrix\}\
\\] is a vector in $R^2$ while \
\
\\[\
\\begin\{bmatrix\}\
3 \\\\\
-2\\\\\
0\\\\\
\\end\{bmatrix\}\
\\] is a vector in $R^3$.\
\
\\textbf\{Vector space has to be closed under addition and multiplication.\}\
\
\\textbf\{Subspace of $R^2$\} %= vector space of $R^2$.\
For a line to be subspace in $R^2$, the line must pass though origin else it won't satisfy the properties of a vector space.\
\
\\newline\
\\textbf\{Various subspaces of $R^2$\}\
\\begin\{enumerate\}\
    \\item $R^2$ itself. (plane)\
    \\item any line through \\[\
\\begin\{bmatrix\}\
0\\\\\
0\\\\\
\\end\{bmatrix\}\
\\] (L)\
    \\item zero vector (Z)\
\\end\{enumerate\}\
\
\\textbf\{Various subspaces of $R^3$\}\
\\begin\{enumerate\}\
    \\item $R^3$ itself. \
    \\item any line through \\[\
\\begin\{bmatrix\}\
0\\\\\
0\\\\\
0\\\\\
\\end\{bmatrix\}\
\\] (L)\
    \\item any plane through \\[\
\\begin\{bmatrix\}\
0\\\\\
0\\\\\
0\\\\\
\\end\{bmatrix\}\
\\]\
    \\item zero vector (Z)\
    \
\\end\{enumerate\}\
\\textbf\{How to create a subspace from a matrix?\}\\newline\
Let A  = \\[\
\\begin\{bmatrix\}\
1 & 3\\\\\
2& 3\\\\\
4 & 1\\\\\
\\end\{bmatrix\}\
\\]\
All the linear combinations of columns of A form a subspace. Called as \\textbf\{column subspace\}, C(A). This subspace is a plane passing through the origin. If the 2 columns of A were lying on the same line, then we would have got that line as subspace instead of plane.\
Since columns of A are vectors in $R^3$, the vectors in subspace obtained from it are vectors in $R^3$.\
\\newline\
Let's say we have 5 vectors in $R^\{10\}$. What is the subspace formed by these vectors? \\newline\
We won't get $R^5$ because these vectors have 10 components, not 5. We might get a 5-d flat thing passing through origin. If those 5 vectors lie on the same line, then we will only get a line passing through origin.\
\\end\{document\}\
}
